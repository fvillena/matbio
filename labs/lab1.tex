\documentclass{article}
\usepackage[utf8]{inputenc}

\title{Laboratorio 1: Matrices}
\author{Fabián Villena}
\date{Abril 2024}

\begin{document}

\maketitle

\section{Problema}

Usted es el científico de datos de un hospital en el cual se está realizando un análisis descriptivo de un grupo de pacientes. Las enfermeras y enfermeros capturadores de datos realizaron mediciones antropometricas de cada uno de los individuos y la escribieron a mano en un \textit{Case Report Form} (CRF). Posteriormente un operador traspasó cada CRF a una planilla. A usted le entregan la planilla en un archivo Excel (Tabla \ref{tab:excel}) y se le solicita analizarla. 

Con sus conocimientos del lenguaje de programación \texttt{R} realice las siguientes actividades.

\begin{table}[h]
\centering
\begin{tabular}{lllccc}
\hline
nombre    & apellido  & rut        & altura & peso & circunf\_abdominal \\ \hline
Juan      & Pérez     & 11111111-1 & 1.63   & 77   & 0.65                      \\
Pedro     & Pereira   & 22222222-2 & 1.53   & 78   & 0.88                      \\
Pablo     & Gómez     & 33333333-3 & 1.90   & 91   & 0.83                      \\
Juan      & Romero    & 44444444-4 & 1.80   & 138  & 1.20                      \\
María     & Silva     & 55555555-5 & 1.45   & 52   & 0.63                      \\
Felipe    & Ruiz      & 66666666-6 & 1.80   & 40   & 0.67                      \\
Óscar     & Díaz      & 77777777-7 & 1.71   & 75   & 1.12                      \\
Eliana    & Quiroga   & 88888888-8 & 1.75   & 90   & 0.90                      \\
Camila    & Sosa      & 99999999-9 & 1.68   & 123  & 0.70                      \\
Constanza & Fernández & 00000000-0 & 1.66   & 60   & 0.88                      \\ \hline
\end{tabular}
\caption{Planilla Excel}
\label{tab:excel}
\end{table}

\section{Actividades}

\begin{enumerate}
    \item Construya una matriz que contenga sólo las mediciones antropométricas asociadas a los pacientes.
    \item Construya una matriz que contenga sólo los datos personales de los pacientes.
    \item Calcule un vector que contenga el índice de masa corporal de cada uno de los pacientes.
    \item Haga un subconjunto de la matriz de datos personales de los pacientes que sólo muestre el nombre y apellido de los pacientes.
    \item (opcional) Calcule un vector con valores \texttt{logical} que muestre si el paciente está en el peso normal o no.
\end{enumerate}

\end{document}
