\documentclass{article}
\usepackage[utf8]{inputenc}
\usepackage{hyperref}
\usepackage{attachfile}

\title{Laboratorio 3: Visualización}
\author{Fabián Villena \& Jocelyn Dunstan}
\date{Abril 2022}

\begin{document}

\maketitle

\section{Problema}

La pandemia mundial por la enfermedad producida por el virus SARS-CoV-2 ha sido un problema que ha traído devastadoras consecuencias. Los países han tenido que generar estrategias tendientes a disminuir los contagios del virus a través de la restricción de la movilidad y al mismo tiempo intentar no desacelerar la productividad.

Para poder monitorear el nivel de riesgo de contagiosidad en un momento específico y tomar decisiones sobre la restricción de la movilidad se utiliza la tasa de positividad, la cual está calculada por 

\begin{equation}
\textrm{Tasa de positividad}(t) = \frac{\textrm{Cantidad de pruebas positivas en el día } t}{\textrm{Cantidad de pruebas realizadas en el día } t}.
\end{equation}

Gracias al creciente interés en la ciencia de datos, muchas organizaciones comenzaron a publicar los datos asociados a la pandemia. En el caso de Chile, los Ministerios de Salud y de Ciencia, Tecnología, Conocimiento e Innovación fueron los encargados de liberar estos datos.

En el siguiente se encuentra el Data Product 65 que contiene la tasa de positividad de la prueba RT-PCR para el virus SARS-CoV-2 por comuna.

\begin{center}
    \url{https://raw.githubusercontent.com/MinCiencia/Datos-COVID19/master/output/producto65/PositividadPorComuna_std.csv}
\end{center}

Con sus conocimientos del lenguaje de programación R se le solicita analizar el conjunto de datos y realizar las siguientes visualizaciones.

\section{Visualizaciones solicitadas}

\begin{enumerate}
    \item Tasa de positividad en la comuna de La Cisterna a lo largo de toda la pandemia.
    \item Tasa de positividad en la comuna de La Cisterna a lo largo de toda la pandemia señalando a qué altura está la tasa de positividad del 5 \%.
    \item Cantidad de pruebas positivas y cantidad de pruebas realizadas en la comuna de Santiago Centro a lo largo de la pandemia.
    \item Tasa de positividad en la comuna de Santiago Centro en los inviernos de los años 2020 y 2021.
    \item Última tasa de positividad para 3 comunas.
    \item Última tasa de positividad comunal agregada para las regiones de Arica y Magallanes y la Antártica Chilena (pista: gráfico de cajas)
\end{enumerate}

\end{document}
