\documentclass{article}
\usepackage[utf8]{inputenc}
\usepackage{hyperref}

\usepackage{filecontents}

\begin{filecontents}{jobname.bib}
    @article{Sobar2016,
        doi = {10.1166/asl.2016.7980},
        url = {https://doi.org/10.1166/asl.2016.7980},
        year = {2016},
        month = oct,
        publisher = {American Scientific Publishers},
        volume = {22},
        number = {10},
        pages = {3120--3123},
        author = {Sobar and Rizanda Machmud and Adi Wijaya},
        title = {Behavior Determinant Based Cervical Cancer Early Detection with Machine Learning Algorithm},
        journal = {Advanced Science Letters}
    }
\end{filecontents}

\usepackage[backend=bibtex]{biblatex} 
\addbibresource{jobname.bib}

\title{Laboratorio 4: Inverencia}
\author{Fabián Villena \& Jocelyn Dunstan}
\date{Abril 2022}

\begin{document}

\maketitle

\section{Problema}

Una de las características de la ciencia moderna es la disponibilización por parte de los autories de los datos utilizados en las publicaciones el aseguramiento la reproducibilidad de los resultados y así mejorar la validez externa de las conclusiones. Esto también conlleva ciertos desafíos éticos en el área de la salud debido a la sensibilidad de ciertos datos, desafios los cuales se pueden sobrellevar a través de un proceso de anonimización.

\citeauthor{Sobar2016} analizaron la relación que existe entre ciertos atributos de riesgo obtenidos a través de un cuestionario estándar y el desarrollo de cáncer cervical. Estos autores liberaron su conjunto de datos, el cual se encuentra adjunto.

Se le solicita verificar si existen diferencias estadísticamente significativas entre el grupo enfermo y sano respecto a cada uno de los atributos medidos.

\printbibliography

\end{document}
