\documentclass{article}
\usepackage[utf8]{inputenc}
\usepackage{hyperref}

\title{Laboratorio 2: Data Frames}
\author{Fabián Villena \& Jocelyn Dunstan}
\date{Abril 2022}

\begin{document}

\maketitle

\section{Problema}

En Chile, los datos de defunciones son públicos y están administrados por el Departamento de Estadísticas e Información en Salud (DEIS). Estos conjuntos de datos de defunciones están disponibles en el portal de datos abiertos del sitio web del DEIS\footnote{\url{https://deis.minsal.cl/\#datosabiertos}} junto con otras series de datos públicas.

Se le solicita analizar un conjunto de datos de defunciones en Chile utilizando sus conocimiento del lenguaje de programación \texttt{R}.

Adjunto encontrará los datos de defunciones del año 2018.

\section{Preguntas}

\begin{enumerate}
    \item Calcule la cantidad de personas que fallecieron en todo el año.
    \item Calcule la cantidad de personas que fallecieron en la Región Metropolitana en todo el año.
    \item Calcule la cantidad de personas que fallecieron por \texttt{Influenza [gripe] y neumonía} en el mes de mayo.
    \item Calcule la edad promedio de fallecimiento de las mujeres y de los hombres en todo el año.
    \item Calcule qué mes tuvo la mayor y menor cantidad de fallecidos.
\end{enumerate}

\end{document}
